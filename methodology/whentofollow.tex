\section{When Waterfall Model should be followed}
\vspace{0.2cm}
\begin{itemize}
\item \textbf{Well-Defined and Stable Requirements:} The Waterfall model is suitable when the project requirements are clear, detailed, and unlikely to change significantly during the development process. It works best when the project scope is well-understood from the beginning.
\item \textbf{Sequential and Linear Progression:} If the project can be broken down into distinct and sequential phases, with each phase building upon the previous one, the Waterfall model can provide a structured approach. This model is ideal for projects that require a step-by-step progression.
\item \textbf{Fixed Time and Budget Constraints:} When there are strict time and budget constraints, the Waterfall model can be beneficial. Its linear nature allows for more accurate estimation and planning, enabling better control over resources and schedules.
\item \textbf{Regulatory and Compliance Requirements:} Projects that require thorough documentation, validation, and compliance with regulatory standards often benefit from the Waterfall model. Its emphasis on documentation and a well-defined process aligns well with such requirements.
\item \textbf{Experienced Team and Technology:} When the development team is experienced and familiar with the technology being used, the Waterfall model can be an effective choice. The team's expertise allows for better prediction of project outcomes and reduces the risk of unforeseen challenges.
\end{itemize}